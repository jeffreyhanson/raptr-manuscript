\documentclass{letter}
\usepackage[a4paper, left=2.5cm, right=2.5cm, top=2.5cm, bottom=2.5cm]{geometry}
\usepackage{xcolor}
\usepackage{hyperref}
\hypersetup{
  colorlinks=true,
  urlcolor=blue
}
\signature{\vspace{-20pt} Jeffrey Hanson}
\address{School of Biological Sciences\\The University of Queensland\\Brisbane, QLD, Australia}

\longindentation=-1pt
\begin{document}
\begin{letter}{Department of Animal and Plant Sciences\\University of Sheffield\\Sheffield S10 2TN, UK}
\opening{Dear Professor Rob Freckleton,}

Biodiversity is in crisis. To combat further declines, the most fundamental aim of conservation is to maximize the long-term persistence of biodiversity. To achieve this, conservation actions must preserve biodiversity patterns (eg. populations, species, ecosystems), but also crucially the ecological and evolutionary processes that sustain them. Protected area networks buffer species from gross threatening processes and set the stage for enhanced management interventions. As you are aware, a wealth of data relevant to biodiversity processes has become available to the global scientific community over the last decade (eg. genetic, environmental, and morphological data). However, currently available decision support tools only have a limited capacity to use such data to guide conservation planning.

To begin to fill this gap, our manuscript entitled \textit{"raptr: Representative and Adequate Prioritization Toolkit in R"} describes a new conservation reserve planning tool that explicitly accommodates data pertaining to biodiversity patterns and biodiversity processes. This tool is available as an \texttt{R} package on \textit{The Comprehensive R Archive Network} (\url{https://CRAN.R-project.org/package=raptr}). By applying the tool to a simulation study and two case studies, we showcase its behavior and compare it with conventional methods for reserve selection. In each of these studies, we found that explicitly targeting a representative sample of a species--for instance in terms of its genetic variation to conserve evolutionary processes--resulted in a different set of conservation priorities.

We chose to publish our work in \textit{Methods in Ecology and Evolution} because of its focus on novel statistical and computational methodologies. We desire to communicate our findings to the global community of ecologists and conservation biologists that follow this journal. The content of our manuscript has not been published elsewhere has not been published elsewhere, nor has it been submitted for publication to another journal. All authors approved the manuscript and this submission. In support of the journal's commitment to reproducible research, we have stored all data, code, and results in an online repository to permit replication and validation of this study (\url{https://www.github.com/jeffreyhanson/raptr-manuscript}). 

\closing{Thank you for your consideration,}

\end{letter}
\end{document}
