% load packages
\usepackage{amsmath,amsfonts,float,makecell,titletoc,titlesec,lineno,booktabs}
\usepackage[T1]{fontenc}
\usepackage{lmodern}
\usepackage[utf8]{inputenc}
\usepackage[doublespacing]{setspace}

% format captions
\usepackage[labelfont={small,bf}, labelsep=space, font={small}]{caption}

% line numbers
\linenumbers

% allow breaks in equations
\allowdisplaybreaks

% format abstract
\renewcommand{\abstractname}{Summary}
\renewenvironment{abstract}
 {\small
  \begin{center}
  \bfseries \abstractname\vspace{-.5em}\vspace{0pt}
  \end{center}
  \list{} {%
   \setlength{\leftmargin}{2mm}
   \setlength{\rightmargin}{\leftmargin}%
  }%
  \item\relax}
{\endlist}

% format section headers
\titleformat*{\section}{\Large\bfseries}
\titleformat{\subsection}[display]
	{\large\sffamily\lsstyle}
	{\subsectiontitlename\ \thesubsection}
	{0.5em}{}
\titleformat*{\subsubsection}{\large\itshape}

% make figures static
\let\origfigure\figure
\let\endorigfigure\endfigure
\renewenvironment{figure}[1][2] {
	\expandafter\origfigure\expandafter[H]
} {
	\endorigfigure
}

% define struts for tables
\newcommand\T{\rule{0pt}{2.6ex}} % top strut
\newcommand\B{\rule[-1.2ex]{0pt}{0pt}} % bottom strut

% define command to put new lines in table cells
%\newcommand{\specialcell}[2][c]{%
%  \begin{tabular}[#1]{@{}c@{}}#2\end{tabular}}

\newcommand{\noopsort}[2]{#2}